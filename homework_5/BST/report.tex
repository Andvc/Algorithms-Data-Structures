\documentclass{article}
\usepackage{amsmath}
\usepackage{graphicx}
\usepackage{hyperref}
\usepackage{ctex} % 支持中文

\title{homework_4}
\author{万成-322014028}
\date{\today}

\begin{document}

\maketitle

\section{引言}
描述对二叉搜索树(BST)实现所做的修改和对 `remove` 函数的优化:通过避免不必要的节点值复制和递归删除来提高效率。

\section{修改细节}
在原始实现中,当要删除的节点有两个子节点时,`remove` 函数复制了要删除节点的值,这导致了效率低下。为了解决这个问题,函数的结构被重构,利用指针操作和一个辅助函数 `detachMin`。

\subsection{detachMin 函数}
`detachMin` 函数负责查找并移除子树中的最小节点。这样可以更高效地替换被删除节点,而无需复制其值。

\begin{verbatim}
BinaryNode *detachMin(BinaryNode *&t) {
    if (t == nullptr) {
        return nullptr;
    }
    if (t->left == nullptr) {
        BinaryNode *minNode = t;
        t = t->right;  // 调整指针
        return minNode;  // 返回被移除的最小节点
    }
    return detachMin(t->left);
}
\end{verbatim}

\subsection{更新后的 remove 函数}
`remove` 函数被修改为在节点有两个子节点的情况下使用 `detachMin`:

\begin{verbatim}
void remove(const Comparable &x, BinaryNode *&t) {
    if (t == nullptr) return;

    if (x < t->element) {
        remove(x, t->left);
    } else if (x > t->element) {
        remove(x, t->right);
    } else {
        if (t->left != nullptr && t->right != nullptr) {
            BinaryNode *minNode = detachMin(t->right);
            t->element = minNode->element;
            delete minNode;  // 删除被移除的节点
        } else {
            BinaryNode *oldNode = t;
            t = (t->left != nullptr) ? t->left : t->right;
            delete oldNode;
        }
    }
}
\end{verbatim}

这种实现避免了不必要的复制和递归,显著提高了删除节点时的性能。

\section{测试方法}
测试过程包括向 BST 中插入元素,执行各种操作,并验证输出。测试的关键操作包括:

\begin{itemize}
    \item 插入多个元素
    \item 获取最小和最大元素
    \item 检查特定元素的存在性
    \item 在不同情况下删除节点(包含两个子节点、一个子节点和叶节点)
    \item 清空树
    \item 测试拷贝和移动语义
    \item 在操作空树时的异常处理
\end{itemize}

\section{测试结果}
测试输出确认了操作的正确性。以下是一些关键结果:

\subsection{初始树}
插入元素后的初始树结构正确显示:

\begin{verbatim}
初始树:
3
5
7
10
12
15
18
\end{verbatim}

\subsection{删除后的树}
每次删除后的树结构也如预期打印出来,确认了适当节点的删除:

- 删除 7 后:
\begin{verbatim}
删除 7 后的树:
3
5
10
12
15
18
\end{verbatim}

- 删除 10 后(有两个子节点):
\begin{verbatim}
删除 10 后的树:
3
5
12
15
18
\end{verbatim}

\subsection{最终状态}
在所有操作后,确认树为空的最终状态:

\begin{verbatim}
清空后的树:
空树
树是否为空? 是
\end{verbatim}

\section{结论}
对二叉搜索树实现中 `remove` 函数的修改提高了效率,避免了不必要的复制和递归删除。所进行的测试验证了 BST 操作的正确性和鲁棒性,确保数据结构在各种情况下保持其完整性。

\end{document}
