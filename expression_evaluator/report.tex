\documentclass[12pt]{article}
\usepackage{ctex} % 支持中文
\usepackage{geometry}
\usepackage{listings}
\usepackage{xcolor}

\geometry{a4paper, margin=1in}

\title{四则运算homework}
\author{万成-3220104028}
\date{\today}

\begin{document}

\maketitle

\section{一句话简介}

一个四则运算表达式求值程序,能识别加减乘除,重括号,小数,负数,科学计数法(加分项)


\section{设计与实现}


\subsection{主程序(main.cpp)}

主程序负责处理用户输入、输出结果以及运行预定义的测试用例。它调用表达式求值类的接口,获取计算结果或判断表达式是否合法。

\subsection{表达式求值类(expression\_evaluator.h)}

该类包含了表达式解析和计算的核心逻辑,主要包括以下步骤:

\begin{enumerate}
    \item \textbf{分词(Tokenization)}:将输入的字符串表达式分解为数字、运算符和括号等标记。
    \item \textbf{中缀转后缀(Shunting Yard Algorithm)}:利用栈将中缀表达式转换为后缀表达式,以便于计算。
    \item \textbf{后缀表达式计算}:根据后缀表达式计算最终结果。
\end{enumerate}

此外,程序还包含了错误检测机制,用于识别非法表达式,并在检测到错误时返回“ILLEGAL”。

\section{测试与验证}

为了确保程序的正确性和鲁棒性,设计了多组测试用例,包括:

\subsection{合法表达式}

\begin{itemize}
    \item 简单运算:$1+2 = 3$
    \item 括号运算:$(1+2)*3 = 9$
    \item 负数和科学计数法:$-1 + 2e2 = 199$
\end{itemize}

\subsection{非法表达式}

\begin{itemize}
    \item 括号不匹配:$(1+2$ 或 $1+2)$
    \item 连续运算符:$1++2.1$ 或 $1 +* 2$
    \item 除数为零:$1/0$
    \item 科学计数法错误:$1 + 2e$ 或 $1 + 2e+$
    \item 空表达式或仅有空格
\end{itemize}

测试完证明了可靠性。

\section{总结}



\end{document}
