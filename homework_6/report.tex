\documentclass[12pt,a4paper]{article}
\usepackage{amsmath}
\usepackage{ctex}  % 用于支持中文

\title{ \texttt{remove} 函数的实现}
\author{}
\date{}

\begin{document}

\maketitle



\texttt{remove} 函数的定义:

\begin{verbatim}
void remove(const Comparable &x) {
    remove(x, root);
}
\end{verbatim}

函数调用了一个递归辅助函数 \texttt{remove} 来操作树的根节点。

\section{函数 \texttt{remove} 实现}

实际的删除操作是由辅助函数 \texttt{remove} 完成的:

\begin{verbatim}
void remove(const Comparable &x, BinaryNode *& t) {
    if (t == nullptr) return;

    if (x < t->element) {
        remove(x, t->left);
    } else if (x > t->element) {
        remove(x, t->right);
    } else if (t->left != nullptr && t->right != nullptr) {
        BinaryNode *minNode = detachMin(t->right);
        t->element = minNode->element;
        delete minNode;
    } else {
        BinaryNode *oldNode = t;
        t = (t->left != nullptr) ? t->left : t->right;
        delete oldNode;
    }
    balance(t);
}
\end{verbatim}

工作流程:
\begin{enumerate}
    \item \textbf{基本情况:} 如果当前节点 \texttt{t} 为 \texttt{nullptr},表示树为空,函数直接返回。
    \item \textbf{查找节点:} 如果值 \texttt{x} 小于当前节点的元素,则递归地在左子树中查找并删除节点;如果 \texttt{x} 大于当前节点的元素,则递归地在右子树中查找并删除节点。
    \item \textbf{具有两个子节点的节点:} 如果待删除的节点有两个子节点,我们通过 \texttt{detachMin} 获取右子树中的最小节点,将当前节点的值替换为最小节点的值,然后删除最小节点。
    \item \textbf{具有一个或没有子节点的节点:} 如果节点只有一个子节点或没有子节点,直接用其子节点(如果有的话)替换当前节点,并删除当前节点。
\end{enumerate}

\section{平衡树}

在删除节点之后,树可能会失去平衡。为了确保树的平衡性,调用 \texttt{balance} 函数,该函数根据树的高度调整节点的位置,通过旋转节点来恢复 AVL 树的平衡性。


\end{document}