\documentclass[UTF8]{ctexart}
\usepackage{geometry, CJKutf8}
\geometry{margin=1.5cm, vmargin={0pt,1cm}}
\setlength{\topmargin}{-1cm}
\setlength{\paperheight}{29.7cm}
\setlength{\textheight}{25.3cm}

% useful packages.
\usepackage{amsfonts}
\usepackage{amsmath}
\usepackage{amssymb}
\usepackage{amsthm}
\usepackage{enumerate}
\usepackage{graphicx}
\usepackage{multicol}
\usepackage{fancyhdr}
\usepackage{layout}
\usepackage{listings}
\usepackage{float, caption}

\lstset{
    basicstyle=\ttfamily, basewidth=0.5em
}

% some common command
\newcommand{\dif}{\mathrm{d}}
\newcommand{\avg}[1]{\left\langle #1 \right\rangle}
\newcommand{\difFrac}[2]{\frac{\dif #1}{\dif #2}}
\newcommand{\pdfFrac}[2]{\frac{\partial #1}{\partial #2}}
\newcommand{\OFL}{\mathrm{OFL}}
\newcommand{\UFL}{\mathrm{UFL}}
\newcommand{\fl}{\mathrm{fl}}
\newcommand{\op}{\odot}
\newcommand{\Eabs}{E_{\mathrm{abs}}}
\newcommand{\Erel}{E_{\mathrm{rel}}}

\begin{document}

\pagestyle{fancy}
\fancyhead{}
\lhead{万成, 3220104028}
\chead{数据结构与算法第四次作业}
\rhead{Oct.20th, 2024}

\section{测试程序的设计思路}

为了全面测试 \texttt{List} 类的各项功能,我设计了多个测试用例,覆盖了列表的基本操作以及拷贝和移动语义。具体设计思路如下:

\begin{enumerate}
    \item \textbf{初始化测试:} 首先创建一个空的整数列表,验证其是否为空以及大小是否为零。
    \item \textbf{插入操作测试:} 
    \begin{itemize}
        \item 使用 \texttt{push\_back} 在列表尾部插入元素,测试尾部插入功能。
        \item 使用 \texttt{push\_front} 在列表头部插入元素,测试头部插入功能。
    \end{itemize}
    \item \textbf{访问操作测试:} 通过 \texttt{front} 和 \texttt{back} 方法访问首尾元素,确保访问功能正确。
    \item \textbf{删除操作测试:} 
    \begin{itemize}
        \item 使用 \texttt{pop\_front} 删除头部元素,测试头部删除功能。
        \item 使用 \texttt{pop\_back} 删除尾部元素,测试尾部删除功能。
    \end{itemize}
    \item \textbf{插入与删除特定位置测试:} 
    \begin{itemize}
        \item 使用 \texttt{insert} 方法在指定位置插入元素。
        \item 使用 \texttt{erase} 方法删除指定位置的元素。
    \end{itemize}
    \item \textbf{清空测试:} 使用 \texttt{clear} 方法清空列表,验证列表是否为空。
    \item \textbf{初始化列表构造函数测试:} 创建一个包含字符串的列表,测试初始化列表构造函数。
    \item \textbf{拷贝构造函数与赋值运算符测试:} 
    \begin{itemize}
        \item 使用拷贝构造函数创建列表的副本。
        \item 使用赋值运算符将一个列表赋值给另一个列表。
    \end{itemize}
    \item \textbf{移动构造函数与移动赋值运算符测试:} 
    \begin{itemize}
        \item 使用移动构造函数将一个列表的数据移动到另一个列表。
        \item 使用移动赋值运算符将一个列表的数据移动到另一个列表。
    \end{itemize}
    \item \textbf{内存泄漏检测:} 使用 \texttt{valgrind} 工具检测程序在运行过程中是否存在内存泄漏。
\end{enumerate}

通过上述测试用例,确保 \texttt{List} 类的各项功能在不同场景下均能正确运行。

\section{测试的结果}

所有设计的测试用例均成功通过,具体结果如下:

\begin{enumerate}
    \item \textbf{初始化测试:} 成功创建一个空的整数列表,\texttt{empty()} 返回 \texttt{true},\texttt{size()} 返回 0。
    \item \textbf{插入操作测试:} 
    \begin{itemize}
        \item 使用 \texttt{push\_back} 插入 10, 20, 30 后,列表顺序为 10, 20, 30。
        \item 使用 \texttt{push\_front} 插入 5, 2 后,列表顺序为 2, 5, 10, 20, 30。
    \end{itemize}
    \item \textbf{访问操作测试:} 
    \begin{itemize}
        \item \texttt{front()} 返回 2。
        \item \texttt{back()} 返回 30。
    \end{itemize}
    \item \textbf{删除操作测试:} 
    \begin{itemize}
        \item 使用 \texttt{pop\_front} 删除首元素后,列表为 5, 10, 20, 30。
        \item 使用 \texttt{pop\_back} 删除尾元素后,列表为 5, 10, 20。
    \end{itemize}
    \item \textbf{插入与删除特定位置测试:} 
    \begin{itemize}
        \item 在第二个位置插入 15 后,列表为 5, 15, 10, 20。
        \item 删除第二个位置的元素 (15) 后,列表回到 5, 10, 20。
    \end{itemize}
    \item \textbf{清空测试:} 使用 \texttt{clear} 方法后,列表为空,\texttt{empty()} 返回 \texttt{true},\texttt{size()} 返回 0。
    \item \textbf{初始化列表构造函数测试:} 使用 \texttt{"苹果", "香蕉", "樱桃"} 初始化字符串列表,列表顺序为 苹果, 香蕉, 樱桃。
    \item \textbf{拷贝构造函数与赋值运算符测试:} 
    \begin{itemize}
        \item 拷贝构造函数创建的副本列表包含 苹果, 香蕉, 樱桃。
        \item 赋值运算符赋值后的列表同样包含 苹果, 香蕉, 樱桃。
    \end{itemize}
    \item \textbf{移动构造函数与移动赋值运算符测试:} 
    \begin{itemize}
        \item 移动构造函数移动后的列表包含 苹果, 香蕉, 樱桃,原始列表大小为 0。
        \item 移动赋值运算符移动后的目标列表包含 苹果, 香蕉, 樱桃,源列表在移动后大小为 0。
    \end{itemize}
    \item \textbf{内存泄漏检测:} 使用 \texttt{valgrind} 工具检测程序运行过程中没有发现内存泄漏,所有动态分配的内存均被正确释放。
\end{enumerate}

\section{结论}

感觉没bug,感觉没内存泄露。如果有那确实就是有了,没有的话那就是没有。

\end{document}
